% Preamble
\documentclass[17pt]{article}

% Math related packages.
\usepackage{amsfonts, amsthm, amsmath, amssymb, amstext}
\usepackage{mathtools}
\usepackage{physics}
\usepackage{cancel, textcomp}
\usepackage[mathscr]{euscript}
\usepackage[nointegrals]{wasysym}
\usepackage[left=3cm, right=3cm, top=1.5cm, bottom=1.5cm]{geometry}

\usepackage{esvect}
\usepackage{IEEEtrantools}

% Font related packages.
\usepackage[french]{babel}
\usepackage[unicode]{hyperref}
\usepackage[fontsize = 10pt]{scrextend}
\usepackage[T1]{fontenc}
\usepackage[utf8]{inputenc}
\usepackage[finemath]{kotex}
\usepackage{dhucs-nanumfont}
\usepackage{mathpazo}
\usepackage{FiraMono}
\usepackage{mathrsfs}

% Other
\usepackage{enumerate}
\usepackage[shortlabels]{enumitem}
\usepackage{tabularx}
\usepackage[object=vectorian]{pgfornament}
\usepackage{pgf}
\usepackage{pgfpages}
\usepackage[european,straightvoltages]{circuitikz}
\usepackage{scalerel}
\usepackage{stackengine}


% A more colorful world.
\usepackage{xcolor}

% Display Source Code
\usepackage{listings}
\lstset{
    language=C,
    basicstyle=\small\ttfamily\mdseries,
    numberstyle=\color{gray},
    stringstyle=\color[HTML]{933797},
    commentstyle=\color[HTML]{228B22},
    emph={[2]from,with,import,as,pass,return,and,or,not},
    emphstyle={[2]\color[HTML]{DD52F0}},
    emph={[3]range,format,enumerate,print},
    emphstyle={[3]\color[HTML]{D17032}},
    emph={[4]if,elif,else,for,while,in,def,lambda,int,float,all,len},
    emphstyle={[4]\color{blue}},
    emph={[5]abs},
    emphstyle={[5]\color{black}},
    showstringspaces=false,
    breaklines=true,
    prebreak=\mbox{{\color{gray}\tiny$\searrow$}},
    numbers=left,
    xleftmargin=15pt
}

% OPTIONS
\everymath\expandafter{\the\everymath\displaystyle}

% COMMANDS
\newcommand{\f}[1]{\texttt{#1}}
\newcommand{\sct}[1]{
	\begin{center}
		\Large\textbf{#1}
	\end{center}
}
\newcommand{\subsct}[1]{
	\begin{center}
		\large\textbf{#1}
	\end{center}
}
\newcommand{\inl}[2]{[\![#1, #2]\!]}
\newcommand{\q}[1]{\textbf{#1.}\quad}
\newcommand{\urlsymbol}{\kern1pt\vbox to .5ex{}\raise.10ex\hbox{\pdfliteral{%
    q .8 0 0 .8 0 0 cm
    2.5 5 m 1 j 1 J .8 w
    1 5 l 0 5 0 4 y 0 1 l 0 0 1 0 y 4 0 l 5 0 5 1 y 5 2.5 l S
    3 3 m 6 6 l S 4 6 m 6 6 l 6 4 l S
Q}}\kern5pt}

\newcounter{iloop}
\newcommand\openbigstar[1][0.7]{%
  \scalerel*{%
    \stackinset{c}{-.125pt}{c}{}{\scalebox{#1}{\color{white}{$\bigstar$}}}{%
      $\bigstar$}%
  }{\bigstar}
}
\newcommand{\Stars}[1]{\ensuremath{%
\pgfmathtruncatemacro{\imax}{ifthenelse(int(#1)==#1,#1-1,#1)}%
\pgfmathsetmacro{\xrest}{0.9*(1-#1+\imax)}%
\setcounter{iloop}{0}%
\loop\stepcounter{iloop}\ifnum\value{iloop}<\the\numexpr\imax+1
\bigstar\repeat
\openbigstar[\xrest]%
\setcounter{iloop}{0}%
\loop\stepcounter{iloop}\ifnum\value{iloop}<\the\numexpr5-\imax\relax
\openbigstar[.9]\repeat}}


\def\N{\mathbb N}
\def\Z{\mathbb Z}
\def\Q{\mathbb Q}
\def\R{\mathbb R}
\def\Rpe{\mathbb R_+^*}
\def\C{\mathbb C}
\def\K{\mathbb K}
\def\L{\mathbb L}

\def\P{\mathscr{P}}

\def\pe#1{\left\lfloor #1\right\rfloor}
\def\pd#1{\left\lbrace #1\right\rbrace}
\def\pscal#1#2{\langle#1,#2\rangle}

\def\Ker{\text{Ker}}
\def\ord{\text{ord}}

\def\Vect{\text{Vect}}

\def\ni#1{\|#1\|_{\infty}}


\def\ssi{\Leftrightarrow}
\def\Ssi{\Longleftrightarrow}
\def\implique{\Longrightarrow}

\def\sep{\noindent\makebox[\linewidth]{\rule{\paperwidth}{0.4pt}}}

% CUSTOM TITLE
\makeatletter
\def\@maketitle{%
	\newpage
	%  \null% DELETED
	%  \vskip 2em% DELETED
	\begin{center}%
		\let \footnote \thanks
		{\LARGE\bfseries \@title \par}%
		\vskip 1em%
		{\large
			\lineskip .5em%
			\begin{tabular}[t]{c}%
				\@author
			\end{tabular}\par}%
		\vskip 1em%
		{\large \@date}%
	\end{center}%
	\par
	\vskip -1em}
\makeatother

\title{\huge\bfseries Rappels essentiels et exercices\\ Séance 0}
\begin{document}
\maketitle

Comme prévu, ce document regroupe règles de calcul et certaines formules essentielles qu'il impératif de connaître. En vous rendant à la toute fin, quelques exercices d'entraînement sont proposés.

\begin{center}
	\LARGE\bfseries Identités remarquables
\end{center}
\vspace{1cm}
Dans ce qui suit, $a$ et $b$ sont des réels.\\
\begin{center}
\begin{tabular}{ c c c }
	$\boxed{(a+b)^2=a^2+2ab+b^2}$ & $\boxed{(a-b)^2=a^2-2ab+b^2}$ & $\boxed{a^2-b^2=(a+b)(a-b)}$\\\\
	$\boxed{(a+b)^3=a^3+3a^2b+3ab^2+b^3}$ &  & $\boxed{(a-b)^3=a^3-3a^2b+3ab^2-b^3}$ 
\end{tabular}
\end{center}

\begin{center}
	\LARGE\bfseries Puissances
\end{center}
\vspace{1cm}
Dans ce qui suit, $a$ et $b$ sont des réels, $b$ non nul, $n$ et $m$ sont des entiers.\\
\begin{center}
\begin{tabular}{ c c c }
	$\boxed{a^na^m=a^{n+m}}$ & $\boxed{\frac{a^n}{a^m}=a^{n-m}}$ & $\boxed{a^nb^n=(ab)^n}$\\\\
	$\boxed{\frac{a^n}{b^n}=\left(\frac ab\right)^n}$ & $\boxed{a^{nm}=(a^n)^m}$ & $\boxed{(-a)^n=(-1)^na^n}$  
\end{tabular}
\end{center}
\textbf{Attention}\quad $a^{n^2}\neq (a^n)^2$.

\begin{center}
	\LARGE\bfseries Fractions
\end{center}
\vspace{1cm}
Dans ce qui suit, $a,b,c,d$ sont des réels non nuls.
\begin{center}
\begin{tabular}{ c c c }
	$\boxed{\frac ab+\frac cd=\frac{ad+cb}{bd}}$ & $\boxed{\frac ab-\frac cd=\frac{ad-cb}{bd}}$ & $\boxed{\frac{a+b}c=\frac ab+\frac bc}$\\\\
	$\boxed{\frac ab\times\frac cd=\frac{ac}{bd}}$ & $\boxed{\frac{\frac ab}{\frac cd}=\frac{ad}{bc}}$ & $\boxed{\frac aa=1}$\\\\
	$\boxed{-\frac ab=\frac{-a}b=\frac a{-b}}$ & & $\boxed{\frac ab=\frac cd\ssi ad = bc}$
\end{tabular}
\end{center}

\newpage

\begin{center}
	\LARGE\bfseries Exercices
\end{center}
Il est important de consolider ses capacités de factorisation et de bien mener ses calculs littéraux lorsqu'ils se présentent. C'est pour cela que je vais glisser, à plusieurs reprises, des exercices où il va faloir prendre des initiatives. Ne pas oublier de penser aux identités remarquables.

\subsection*{Fractions}
\begin{enumerate}
	\item Simplifier les fractions suivantes : $\frac{32}{40}$;\quad $8^3\times\frac1{4^2}$
	\item Écrire sous forme d'une fraction irréductible : $\frac24-\frac13$;\quad $\frac23-0,2$;\quad $\frac{-\frac2{15}}{-\frac65}$
	\item Écrire sous forme d'une fraction irréductible : $\frac{1 978\times1 979+1 980\times21+1958}{1 980\times1 979−1 978\times1 979}$
\end{enumerate}

\subsection*{Puissances}
\begin{enumerate}
	\item Donner dans chaque cas le résultat sous la forme d'une puissance de $10$ :
	\begin{center}
		\begin{tabular}{p{6cm} p{6cm}}
			a)\quad $10^5\times 10^3$ & b)\quad $\frac{(10^5\times 10^{-3})^5}{(10^{-5}\times 10^3)^{-3}}$\\\\
			c)\quad $(10^5)^3$ & d)\quad $\frac{10^{-5}}{10^{-3}}$
		\end{tabular}
	\end{center}
	\item Dans chaque cas, donner ler résultat sous la forme $a^n$ où $a$ et $n$ sont deux entiers.
	\begin{center}
		\begin{tabular}{p{6cm} p{6cm}}
			a)\quad $3^4\times 5^4$ & b)\quad $\frac{6^5}{2^5}$\\\\
			c)\quad $(5^3)^{-2}$ & d)\quad $(-7)^3\times (-7)^5$
		\end{tabular}
	\end{center}
\end{enumerate}

\subsection*{Calcul littéral}
\begin{enumerate}
	\item Développer, réduire et ordonner selon les puissances de $x$ l'expression \[(x^2+x+1)(x^2-x+1)\]
	\item Factoriser $xy+x+y+1$.
	\item Factoriser $25-(10x+3)^2$.
	\item Factoriser $(x+y)^2-z^2$.
\end{enumerate}
\end{document}